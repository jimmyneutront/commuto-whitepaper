%! Author = jimmyt
%! Date = 1/15/23

% Preamble
\documentclass[11pt]{article}

% Packages
\usepackage{amsmath}

\author{jimmyt}

% Document
\begin{document}

    \title{Commuto}
    \maketitle

    \begin{abstract}
        Commuto is a collection of software allowing private, noncustodial, censorship resistant
        exchange of national currencies and tokens adopting the ERC20\cite{ERC20} standard on
        Ethereum Virtual Machine\cite{Ethereum} compatible blockchains.
        The Commuto Protocol's name comes from the Latin word ``commuto'', meaning ``I exchange'' or
        ``I barter.''
    \end{abstract}

    \section*{Introduction}

        Commuto is primarily composed of two software components: a set of smart contracts deployed
        on an EVM-compatible blockchain, and a set of applications that can interact both with said
        on-chain contracts as well as other Commuto users.
        The smart contracts will be referred to henceforth as ``Commuto Core'' and said applications
        will be referred to as ``Commuto Interfaces''.
        An intention expressed by a Commuto user to buy or sell a particular ERC20 token in exchange
        for one or more particular national currency is known as an ``offer''.
        An exchange of national currency and ERC20 tokens between two users is known as a ``swap''.

\end{document}