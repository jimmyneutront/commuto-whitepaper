%! Author = jimmyt
%! Date = 1/15/23

% Preamble
\documentclass[11pt]{article}

% Packages
\usepackage{amsmath}

\author{jimmyt}

% Document
\begin{document}

    \title{Commuto}
    \maketitle

    \begin{abstract}
    Commuto is a collection of software allowing private, noncustodial, censorship resistant
    exchange of national currencies and tokens adopting the ERC20\cite{ERC20} standard on
    Ethereum Virtual Machine\cite{Ethereum} compatible blockchains.
    The Commuto Protocol's name comes from the Latin word ``commuto'', meaning ``I exchange'' or ``I
    barter.''
    \end{abstract}

    \section*{Introduction}

    Commuto is primarily composed of two software components: a set of smart contracts deployed on
    an EVM-compatible blockchain, and a set of applications that can interact both with said
    on-chain contracts as well as other Commuto users.
    The smart contracts will be referred to henceforth as ``Commuto Core'' and said applications
    will be referred to as ``Commuto Interfaces''.
    An intention expressed by a Commuto user to buy or sell a particular ERC20 token in exchange for
    one or more particular national currency is known as an ``offer''.
    An exchange of national currency and ERC20 tokens between two users is known as a ``swap''.

    Because Commuto allows users to exchange national currency, certain pieces of private
    information (such as bank account details and addresses) must be exchanged between users.
    Additionally, users should be able to communicate with each other in a convenient manner, in
    order to resolve any problems that may arise during the swap process.
    While an EVM blockchain (on which Commuto Core contracts are deployed could) theoretically
    be used to exchange such information by storing it (even if only temporarily) on-chain, the
    relatively high cost of storage on such blockchains makes this approach unfeasible.
    Additionally, many users are likely accustomed to instant messaging applications that
    deliver messages in less than a second.
    The relatively longer time required to incorporate a new transaction into a such blockchain is
    yet another reason why an EVM-blockchain-based communication system between users is not
    practical.
    Therefore, Commuto Interfaces use Matrix\cite{Matrix} to reliably exchange information in a
    decentralized, censorship-resistant manner.
    Matrix is a network of nodes running software conforming to the Matrix
    Specification\cite{MatrixSpec}, which defines a set of open APIs for decentralised
    communication, suitable for securely publishing, persisting and subscribing to data over a
    global open federation of servers with no single point of control.
    A Matrix Room is a directed acyclic graph of Events, the ordering of which is the chronological
    ordering of Events in the room.
    Any node may maintain a copy of any such Room, and nods may add new elements to Rooms.
    Events are simply JSON objects with zero or more ``parent'' events, which are chronological
    predecessors in the event graph.
    Nodes use state resolution algorithms and communicate with one another using federation
    algorithms (as defined in the specification) in order to maintain persistent,
    eventually-consistent synchronization of Room state across all nodes in the network.
    Therefore, no single node (referred to in the Matrix Specification and henceforth in this paper
    as a ``homeserver'') has control over any given Room.
    Thus, due to its open, flexible, decentralized, censorship-resistant nature, Commuto uses Matrix
    to exchange data between Commuto Interfaces.

\end{document}