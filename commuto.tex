%! Author = jimmyt
%! Date = 1/15/23

% Preamble
\documentclass[11pt]{article}

% Packages
\usepackage{amsmath}
\usepackage[margin=1in]{geometry}
\usepackage{longtable}
\usepackage{multirow}
\usepackage{tikz}

%\usepackage{showframe}

\usetikzlibrary{positioning}

\author{jimmyt}

% Document
\begin{document}

    \title{Commuto}
    \maketitle

    \begin{abstract}
    Commuto is a collection of software allowing private, noncustodial, censorship resistant
    exchange of national currencies and tokens adopting the ERC20\cite{ERC20} standard on
    Ethereum Virtual Machine\cite{Ethereum} compatible blockchains.
    The Commuto Protocol's name comes from the Latin word ``commuto'', meaning ``I exchange'' or ``I
    barter.''
    \end{abstract}

    \section*{Introduction}

    Commuto is primarily composed of two software components: a set of smart contracts deployed on
    an EVM-compatible blockchain, and a set of applications that can interact both with said
    on-chain contracts as well as other Commuto users.
    The smart contracts will be referred to henceforth as ``Commuto Core'' and said applications
    will be referred to as ``Commuto Interfaces''.
    An intention expressed by a Commuto user to buy or sell a particular ERC20 token in exchange for
    one or more particular national currency is known as an ``offer''.
    An exchange of national currency and ERC20 tokens between two users is known as a ``swap''.

    Because Commuto allows users to exchange national currency, certain pieces of private
    information (such as bank account details and addresses) must be exchanged between users.
    Additionally, users should be able to communicate with each other in a convenient manner, in
    order to resolve any problems that may arise during the swap process.
    While an EVM blockchain (on which Commuto Core contracts are deployed could) theoretically
    be used to exchange such information by storing it (even if only temporarily) on-chain, the
    relatively high cost of storage on such blockchains makes this approach unfeasible.
    Additionally, many users are likely accustomed to instant messaging applications that
    deliver messages in less than a second.
    The relatively longer time required to incorporate a new transaction into a such blockchain is
    yet another reason why an EVM-blockchain-based communication system between users is not
    practical.
    Therefore, Commuto Interfaces use Matrix\cite{Matrix} to reliably exchange information in a
    decentralized, censorship-resistant manner.
    Matrix is a network of nodes running software conforming to the Matrix
    Specification\cite{MatrixSpec}, which defines a set of open APIs for decentralised
    communication, suitable for securely publishing, persisting and subscribing to data over a
    global open federation of servers with no single point of control.
    A Matrix Room is a directed acyclic graph of Events, the ordering of which is the chronological
    ordering of Events in the room.
    Any node may maintain a copy of any such Room, and nods may add new elements to Rooms.
    Events are simply JSON objects with zero or more ``parent'' events, which are chronological
    predecessors in the event graph.
    Nodes use state resolution algorithms and communicate with one another using federation
    algorithms (as defined in the specification) in order to maintain persistent,
    eventually-consistent synchronization of Room state across all nodes in the network.
    Therefore, no single node (referred to in the Matrix Specification and henceforth in this paper
    as a ``homeserver'') has control over any given Room.
    Thus, due to its open, flexible, decentralized, censorship-resistant nature, Commuto uses Matrix
    to exchange data between Commuto Interfaces.

    \section*{Commuto Core}

    Commuto Core is composed of smart contracts that enable swaps between offer makers and offer
    takers, and also allow the resolution of disputes between makers and takers.
    (For example a dispute may arise when a token buyer claims to have sent payment to the seller,
    but the seller claims that they have not received this payment.)
    We begin by considering the operations of the CommutoSwap smart contract, which enables the
    swapping process.
    Subsequently, we explain the functionality of the contracts allowing dispute resolution,
    governance, and the distribution of service fees collected by CommutoSwap.
    Smart contract code snippets are written in Solidity\cite{Solidity}.
    For readability, we include parameter names in function signatures, even though actual EVM smart
    contract function signatures do not include parameter names.

    \subsection*{CommutoSwap}

    The CommutoSwap smart contract can be best understood by considering the steps of the swap
    process, so we describe it in this context.
    Note that, as described in the introduction, the swap process includes the exchange of
    information via Matrix.
    However, because this section focuses on the operations of CommutoSwap, we temporarily omit the
    details of off-blockchain communication, and instead describe them in a later section.
    Throughout this document, we use the symbol ERC to refer to any ERC20 token, and a the symbol
    CUR to refer to any national currency.
    Sellers have ERC and want CUR, and buyers have CUR and want ERC.

    \subsubsection*{Opening an Offer}

    An offer maker opens an Offer by calling the \verb|openOffer| function.
    This function has the following signature:
    \begin{verbatim}
    openOffer(bytes16 offerID, Offer newOffer)
    \end{verbatim}
    where \verb|offerID| is a Version-4 UUID\cite{UUID}, and \verb|newOffer| is an \verb|Offer|
    struct that is defined as follows:
    \begin{verbatim}
    struct Offer {
        bool isCreated
        bool isTaken
        address maker
        bytes interfaceId
        address stablecoin
        uint256 amountLowerBound
        uint256 amountUpperBound
        uint256 securityDepositAmount
        uint256 serviceFeeRate
        SwapDirection direction
        bytes[] settlementMethods
        uint256 protocolVersion }
    \end{verbatim}
    and \verb|SwapDirection| is an \verb|enum| that is defined as follows:
    \begin{verbatim}
    enum SwapDirection {
        BUY
        SELL
    }
    \end{verbatim}
    A \verb|SwapDirection.BUY| value indicates that the maker of the Offer is a buyer, as previously
    defined, and \verb|SwapDirection.SELL| is similar.
    The properties of the \verb|Offer| struct are as follows: \\

    \begin{longtable}[p]{ |p{2.5cm}|p{4cm}|p{7cm}| }
    \hline
    \multicolumn{3}{|c|}{Offer} \\
    \hline
    Property Type & Property Name & Description \\
    \hline
    bool & isCreated & Used by contract code to check for offer existence, will be set to true
    by \verb|openOffer|. \\
    bool & isTaken & Used by contract code to check whether an offer is taken, will be set to
    false by \verb|openOffer|. \\
    address & maker & The maker's address, will be sent to msg.sender by \verb|openOffer|. \\
    bytes & interfaceId & An array of bytes that should be placed in the ``recipient'' field of any
    message sent to the maker of this offer via Matrix. \\
    address & stablecoin & The contract address of the token that the offer maker is offering to
    swap. \\
    uint256 & amountLowerBound & The lower bound on the range of token amounts that the maker is
    willing to swap. \\
    uint256 & amountUpperBound & The upper bound on the range of token amounts that the maker is
    willing to swap. \\
    uint256 & securityDepositAmount & The token amount to be used as a security deposit, which must
    not be less than ten percent of amountUpperBound. \\
    uint256 & serviceFeeRate & The percentage times 100 of the amount of swapped tokens that the
    maker and taker must pay as a service fee. \\
    SwapDirection & direction & Indicates whether the maker is offering to buy tokens or sell
    tokens. \\
    bytes[] & settlementMethods & An array of \verb|bytes|, each one representing a method by which
    the maker is willing to send/receive payment for tokens. \\
    uint256 & protocolVersion & A number optionally describing which version of the Commuto
    Interface software was used to open this offer. \\
        \hline
    \end{longtable}



\end{document}